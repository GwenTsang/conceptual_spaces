\documentclass{article}
\usepackage{graphicx} % Required for inserting images
\usepackage{stmaryrd}
\usepackage{amssymb}
\title{Membership functions 3}

\begin{document}

\subsection{Graded membership from a superposition of Voronoi diagrams}

\subsubsection*{Discrete (combinatorial) graded membership}
Let \(R=\{r_1,\dots,r_m\}\) be prototypical regions. For each \(i\) let \(P_i=\{p_{i1},\dots,p_{i,n_i}\}\subset\mathbb{R}^2\) be a finite set of candidate prototypes contained in \(r_i\). The set of all ordered configurations (one choice per region) is
\[
\Pi(R):=\prod_{i=1}^m P_i=\{\langle p_1,\dots,p_m\rangle \mid p_i\in P_i\},
\qquad N:=|\Pi(R)|= n^m.
\]
For \(\mathbf p\in\Pi(R)\) denote by \(V_i(\mathbf p)\) the (closed) Voronoi cell of the \(i\)-th prototype in the Voronoi diagram generated by \(\{p_1,\dots,p_m\}\). The combinatorial membership degree of \(x\in\mathbb{R}^2\) to the concept \(C_i\) (associated with \(r_i\)) can be written equivalently in the two forms below:
\[
M_{C_i}(x)
= \frac{1}{N}\sum_{\mathbf p\in\Pi(R)} \mathbf{1}\big(x\in V_i(\mathbf p)\big)
= \frac{\big|\{\mathbf p\in\Pi(R)\mid x\in V_i(\mathbf p)\}\big|}{|\Pi(R)|}.
\]

We can critique this combinatorial function.
First, the function \(M_{C_i}\) is a finite rational-valued step function: for every \(x\), \(M_{C_i}(x)\in\{0,\frac{1}{N},\dots,1\}\). It is piecewise constant (discontinuities occur exactly on the union of Voronoi boundaries across all configurations).

Human graded membership data are typically fit better by smooth curves than by step functions. The combinatorial form is simple and transparent but may produce implausible abrupt jumps when a point crosses an atomic cell boundary.

\subsubsection*{Continuous graded membership (measure-theoretic generalization)}
We now replace the finite prototype sets \(P_i\) by measurable prototypical areas \(r_i\subset\mathbb{R}^2\). The idea is to choose one prototype from each region \(r_i\) according to the Lebesgue measure and measure the proportion of choices that make a given point \(a\) fall in the cell of concept \(C_i\). This yields a continuous graded-membership function that generalizes the discrete counting proportion.

\paragraph{Configuration space and reference measure.}
Define the configuration (completion) space
\[
\Pi(R) := \prod_{i=1}^m r_i \subset (\mathbb{R}^2)^m \cong \mathbb{R}^{2m},
\]
whose generic element we write as \(\mathbf p=(p_1,\dots,p_m)\) with \(p_i\in r_i\). We endow \(\Pi(R)\) with the product Lebesgue measure \(\mu\); thus
\[
\mu(\Pi(R)) = \prod_{i=1}^m \operatorname{vol}(r_i),
\]
which is finite when each prototypical area \(r_i\) is bounded.

\paragraph{Definition of the continuous membership.}
Fix \(a\in\mathbb{R}^2\) and an index \(i\). Let
\[
S_{a,i} := \{\mathbf p\in\Pi(R)\mid \delta(a,p_i) < \delta(a,p_j)\ \text{for all } j\neq i\},
\]
where \(\delta\) denotes Euclidean distance. The measure of \(S_{a,i}\) is
\[
\mu(S_{a,i})=\int_{\Pi(R)} \mathbf{1}_{S_{a,i}}(\mathbf p)\,d\mu(\mathbf p),
\]
and the continuous graded-membership degree is defined by
\[
M_{C_i}(a) \;:=\; \frac{\mu(S_{a,i})}{\mu(\Pi(R))}.
\]

\paragraph{Structure of the defining inequalities.}
The condition \(\delta(a,p_i) < \delta(a,p_j)\) is equivalent (using squared distances) to
\[
\|a-p_i\|^2 < \|a-p_j\|^2
\quad\Longleftrightarrow\quad
2\,a\cdot(p_j-p_i)\;\le\;\|p_j\|^2-\|p_i\|^2,
\]
with strict inequality for tie-breaking. Note carefully:
\begin{itemize}
  \item For fixed \(\mathbf p\) this inequality is linear in the parameter \(a\).
  \item For fixed \(a\) the inequality is \emph{polynomial} (quadratic) in the prototype coordinates \(p_i,p_j\); thus the set \(S_{a,i}\subset\Pi(R)\) is a semi-algebraic set whose boundary is contained in a finite union of algebraic hypersurfaces of degree at most two.
\end{itemize}

\paragraph{Dependence of the volume on \(a\).}
The numerator \(\mu(S_{a,i})\) is the Lebesgue volume of a semi-algebraic set in \(\mathbb{R}^{2m}\) cut out by polynomial inequalities whose coefficients depend polynomially (indeed, linearly or quadratically) on \(a\). Standard facts about parameterized integrals over semi-algebraic domains (or, equivalently, about parametric families of sets with algebraic boundaries) imply that the function \(a\mapsto\mu(S_{a,i})\) is piecewise real-analytic/polynomial on regions of the parameter space where the combinatorial type of the defining inequalities remains constant. Concretely:
\begin{itemize}
  \item As \(a\) varies, the algebraic surfaces bounding \(S_{a,i}\) move continuously; changes in the \emph{combinatorial type} of \(S_{a,i}\) (which inequalities are active, which facets appear/disappear) occur only when \(a\) crosses certain algebraic loci.
  \item On each open region of \(a\)-space where the combinatorial description of \(S_{a,i}\) is constant, \(\mu(S_{a,i})\) is given by an integral of the constant function \(1\) over a domain whose boundary depends polynomially on \(a\); this integral yields a polynomial function of \(a\). Hence \(\mu(S_{a,i})\) is piecewise polynomial in \(a\).
\end{itemize}
Therefore \(M_{C_i}(a)=\mu(S_{a,i})/\mu(\Pi(R))\) is a continuous, piecewise-polynomial function of \(a\) (continuous because small changes of \(a\) produce small changes of the bounding hypersurfaces and thus small changes of volume; non-differentiabilities occur at the algebraic loci where the combinatorial type changes). This provides the sought smoothing of the discrete step function: the continuous measure-theoretic model produces graded-membership functions whose graphs look like curves (piecewise polynomials) rather than steps.

\paragraph{Comments on degree and regularity.}
The precise polynomial degree on each cell depends on the ambient dimension \(2m\) and on the algebraic degree of the defining inequalities (here at most 2). One can expect high-degree pieces in general; for many practical modelling purposes, however, the piecewise-polynomial behaviour is a satisfactory qualitative description explaining why the continuous construction yields smooth-like graded membership.

\subsubsection*{Geometric decomposition induced by superposition (atomic cells, cores, supports)}
We now describe the finer partition of the plane obtained by superimposing Voronoi diagrams across all (discrete) configurations; for the following definitions we revert to the notation where each configuration is an ordered tuple \(\mathbf p\in\Pi(R)\) (this \(\Pi(R)\) may be either the discrete Cartesian product of finite prototype sets or the continuous product of prototype areas depending on context).

\paragraph{Atomic cells} Intuitively, atomic cells are the smallest polygonal regions obtained after drawing every Voronoi diagram (one per configuration) on transparent sheets and layering them: they are the regions on which every Voronoi diagram assigns the same label pattern. These cells are the “steps” of the staircase function, since in each of these cells, the degree of membership $M_{C_i}(x)$ is constant. Formally, let
\[
\mathcal{E} \;:=\; \bigcup_{\mathbf p\in\Pi(R)} \partial\!\Big(\bigcup_{i=1}^m V_i(\mathbf p)\Big)
\]
be the union of all Voronoi boundaries that appear across configurations. Then the co   nnected components of \(\mathbb{R}^2\setminus\mathcal{E}\) are the atomic cells. 


Equivalently, an atomic cell is any non-empty intersection of chosen Voronoi cells, one per configuration:
\[
A \;=\; \bigcap_{\mathbf p\in\Pi(R)} V_{j(\mathbf p)}(\mathbf p),
\]
for some assignment \(j(\mathbf p)\in\{1,\dots,m\}\) for each configuration \(\mathbf p\).

On each atomic cell all indicator functions \(\mathbf{1}(x\in V_i(\mathbf p))\) are constant, hence each combinatorial membership \(M_{C_i}\) is constant on that cell.

\paragraph{Conceptual core.} The conceptual core of \(r_i\) is
\[
K_i \;:=\; \bigcap_{\mathbf p\in\Pi(R)} V_i(\mathbf p).
\]
Using distance notation and allowing possibly infinite/continuous prototype sets, one can characterize \(K_i\) by
\[
x\in K_i \quad\Longleftrightarrow\quad
\sup_{p\in r_i} d(x,p)\;\le\;\inf_{j\ne i}\,\inf_{q\in r_j} d(x,q).
\]
(When each \(r_i\) is finite the \(\sup/\inf\) reduce to \(\max/\min\).)

If \(x\in K_i\) then \(M_{C_i}(x)=1\) (and conversely for the combinatorial finite case).

\paragraph{Support (possibility zone) and transition zone.} The support (possibility zone) of \(r_i\) is
\[
U_i := \bigcup_{\mathbf p\in\Pi(R)} V_i(\mathbf p),
\]
the set of points that can be classified as \(r_i\) for at least one configuration. For two regions \(r_a,r_b\) the transition zone is
\[
F_{ab} := U_a\cap U_b,
\]
i.e. the set of points whose classification depends on the configuration. In the two-region case one has \(K_a^c = U_b\) and \(K_b^c = U_a\), hence \(F_{ab}=(K_a\cup K_b)^c\); for more than two regions the relationships involve unions/intersections over all competing regions and must be handled with care.




\end{document}
