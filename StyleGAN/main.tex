\documentclass[a4paper, 11pt]{article}

% --- PAQUETS ---
\usepackage[utf8]{inputenc}
\usepackage[T1]{fontenc}
\usepackage{lmodern}
\usepackage{amssymb}
\usepackage[french]{babel}
\usepackage[a4paper, landscape, margin=2cm]{geometry} % Passage en format paysage
\usepackage{graphicx} % Nécessaire pour \scalebox
\usepackage{tikz}
\usepackage{amsmath} % Pour \mathbf et \mathbb

% --- LIBRAIRIES TIKZ ---
\usetikzlibrary{
    positioning,        % Pour positionner les nœuds (below=of, etc.)
    shapes.geometric,   % Pour les formes
    arrows.meta,        % Pour personnaliser les flèches
    fit,                % Pour ajuster des boîtes autour d'autres nœuds
    calc,               % Pour les calculs de coordonnées
    decorations.pathreplacing % Pour l'accolade
}

\begin{document}

\begin{figure}[h!]
\centering
% On utilise \scalebox pour réduire la taille globale de 10%
\scalebox{0.9}{% 
\begin{tikzpicture}[
    node distance=0.4cm and 1cm,
    % --- STYLES ---
    gen_block/.style={
        rectangle, draw=black, fill=cyan!10,
        text width=4cm, align=center, minimum height=1.2cm
    },
    final_img/.style={
        rectangle, draw=black, fill=red!20,
        text width=2.5cm, align=center, minimum height=2.2cm
    },
    group_box/.style={
        draw, dashed, line width=1.5pt, inner sep=0.4cm
    },
    group_label/.style={
        align=left, text width=3.5cm
    },
    injection_arrow/.style={
        -{Stealth[length=2.5mm, width=2mm]}, thick, black
    },
    main_arrow/.style={
        -{Stealth[length=3mm]}, thick, black
    }
]

% --- TITRE ---
\node[font=\Large\bfseries] (title) at (6, 2.5) {Injection du style par couche dans StyleGAN3};

% --- BLOCS DE GÉNÉRATION ---
\node[gen_block] (block1) at (0,0) {Bloc de génération \\ 4x4};
\node[gen_block, below=of block1] (block2) {Bloc de génération \\ 8x8};
\node[gen_block, below=of block2] (block3) {Bloc de génération \\ 16x16};

\node[gen_block, below=1cm of block3] (block4) {Bloc de génération \\ 32x32};
\node[gen_block, below=of block4] (block5) {Bloc de génération \\ 64x64};
\node[gen_block, below=of block5] (block6) {Bloc de génération \\ 128x128};

\node[gen_block, below=1cm of block6] (block7) {Bloc de génération \\ 256x256};
\node[gen_block, below=of block7] (block8) {Bloc de génération \\ 512x512};
\node[gen_block, below=of block8] (block9) {Bloc de génération \\ 1024x1024};

% --- VECTEURS DE STYLE ---
\path (block1.north) -- (block9.south) coordinate[midway] (blocks_mid_point);
\node[font=\Large] (w_plus) at (blocks_mid_point -| -8.5, 0) {$\mathbf{w}^+$};

\coordinate (brace_line) at (-6.5, 0);
\draw[decorate, decoration={brace, amplitude=10pt, mirror}, thick]
    (brace_line |- block1.north) -- (brace_line |- block9.south);
\draw[main_arrow] (w_plus.east) -- (brace_line |- blocks_mid_point);

\coordinate (w_label_x) at (-4.5, 0);
\foreach \i in {1,...,9} {
    \node (w\i) at (w_label_x |- block\i.center) {$\mathbf{w}_{\i}$};
}

\foreach \i in {1,...,9} {
    \draw[injection_arrow] (w\i.east) -- (block\i.west);
}

% --- GROUPEMENTS ET SORTIE ---
\node[final_img, right=8cm of block5] (final) {Image finale};

\node[group_box, draw=red, fit=(block1)(block2)(block3)] (group1) {};
\node[group_box, draw=orange, fit=(block4)(block5)(block6)] (group2) {};
\node[group_box, draw=green!50!black, fit=(block7)(block8)(block9)] (group3) {};

\node[group_label, anchor=north west] at ($(group1.north east) + (0.3, 0)$) {Couches haut niveau \\ (Forme, pose)};
\node[group_label, anchor=west, name=label2] at ($(group2.east) + (0.3, 0)$) {Couches moyen niveau \\ (Parties du visage)};
\node[group_label, anchor=north west] at ($(group3.north east) + (0.3, 0)$) {Couches bas niveau \\ (Texture, couleur)};

\draw[main_arrow] (label2.east) -- (final.west);

\end{tikzpicture}
} % Fin du \scalebox

\caption{
    Schéma de l'injection de style par couche dans le générateur de StyleGAN.
    Le vecteur de style étendu $\mathbf{w}^+ \in \mathbf{W}^+$ est décomposé (symbolisé par l'accolade) en sous-vecteurs $\mathbf{w}_i$. Chacun de ces sous-vecteurs est ensuite injecté en entrée du bloc de génération correspondant, comme l'illustrent les flèches horizontales.
    \centering
    \large $\mathbf{w}^+ = [\mathbf{w}_1, \mathbf{w}_2, \dots, \mathbf{w}_n] \quad \text{où} \quad \mathbf{w}_i \in \mathbb{R}^{512}$
}
\label{fig:stylegan_injection}
\end{figure}

\end{document}