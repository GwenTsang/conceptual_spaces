\documentclass{article}
\usepackage[utf8]{inputenc}
\usepackage[T1]{fontenc}
\usepackage{lmodern}
\usepackage[greek,english, french]{babel}
\usepackage{forest}
\usepackage{tikz}
\usepackage{pgfplots}

\usepackage{amssymb}
\usepackage{stmaryrd}
\usepackage{setspace}
\usepackage{amsmath}  % Pour la commande \text{}
\usepackage[a4paper, margin=1.2in]{geometry} % Définit les marges à 1 pouce
\usepackage{caption}
\captionsetup{labelformat=empty}  % Removes "Figure X" from captions

\newcommand{\bolditalic}[1]{\textbf{\textit{#1}}}
\pgfplotsset{compat=1.18}
\begin{document}

\section*{Qu’est-ce qu’un espace conceptuel ?}
\doublespacing  % Définir l'interligne à 2

Chaque espace conceptuel est une représentation géométrique de certains concepts du langage naturel. Il est constitué par des dimensions qui représentent chacune une \textit{qualité} comme la couleur, la hauteur d’un son, la température, etc.\footnote{Gardenfors (2015, sec. 3.5): « Examples of such dimensions are: color, pitch, temperature, weight. »}
L'hypothèse centrale est que le sens peut être représenté par un espace sémantique pouvant être décrit en termes de dimensions, de distances, de régions et d’autres notions géométriques.\footnote{Gärdenfors (2014, sect. 2.1): « A central idea is that the meanings that we use in communication can be described as organized in abstract spatial structures that are expressed in terms of dimensions, distances, regions, and other geometric notions. In addition, I also use some notions from vector algebra. »}
Chaque objet est caractérisé par des valeurs sur ces dimensions, chaque valeur représentant une qualité respective de l'objet.

\subsection*{1.1. Dimensions et domaines}
\doublespacing  % Définir l'interligne à 1.5

Un domaine est un ensemble de dimensions intrinsèquement liées et indissociables. Un domaine est typiquement une \textit{modalité sensorielle}.\footnote{Bechberger, (2023, p.14) : «  quality dimensions are (…) typically assumed to be based on perception ».} 

Par exemple, les couleurs sont un domaine : un stimulus visuel comporte toujours les dimensions de \textit{luminosité} et de \textit{saturation}, perçues de manière conjointe et inséparable. De la même manière, un son est invariablement caractérisé par une \textit{hauteur} et une \textit{intensité}.

\subsection*{1.2. Les propriétés, les concepts et les instances}

Dans un espace conceptuel, une \bolditalic{propriété} est représentée par une \textit{région} dans un domaine unique\footnote{Gardenfors (2000, sec. 3.5) ; (2015, p.4) : « A natural property is a convex region of a domain in a conceptual space ».}. Une propriété correspond à une qualité ou un attribut, souvent exprimé par un adjectif. L'\textit{espace} contenu dans cette région correspond non seulement aux informations sur les membres déjà observés, mais aussi sur les membres \textit{potentiels} non encore rencontrés. Etant définie sur plusieurs dimensions d'un domaine, l’extension de la région indique toutes les combinaisons de valeurs possibles sur ces dimensions pour les membres de la catégorie correspondante. Ainsi, la propriété ROUGE est appréhendée comme une \textit{région} dans le domaine des couleurs. Ces propriétés sont dites \textit{gradées} car elles peuvent prendre des valeurs \textit{continues} dans les intervalles définis par chaque dimension du domaine. Plus précisément, si $\Delta$ est le domaine des couleurs, alors, pour chaque dimension $d_i \in \Delta$, il existe un intervalle de valeurs correspondant à l'extension du rouge sur cette dimension.
\par
Dans un espace conceptuel, chaque \bolditalic{exemplaire} d'une propriété est représenté par un \textit{point} situé à l'intérieur de la région correspondant à cette propriété. Dans le langage courant, l'expression "couleur spécifique" peut faire référence soit à une \textit{région}, soit à un \textit{exemplaire}. Pour éviter cette ambiguïté et distinguer ces deux genres d’emplois, lorsqu’un mot réfère à une \textit{région}, nous l'écrirons en majuscule. Par exemple, cette différence apparaît explicitement dans la phrase : « cette nuance de \bolditalic{rouge} est un \bolditalic{point} dans la \bolditalic{région} du \bolditalic{ROUGE} ». Cette notation permet de distinguer les \textit{petites régions} correspondant à des \textit{propriétés spécifiques} des simples points représentant des exemplaires\footnote{Un exemplaire peut avoir plusieurs instances, le mot "exemplaire" n'a pas de dimension métaphysique ici}. Ainsi, au lieu d’écrire que le carmin \textit{appartient} au ROUGE, on peut employer une relation d'\textit{inclusion} par exemple dans la phrase : « CARMIN est contenue dans ROUGE » pour exprimer que la première est une petite région contenue dans la seconde. La distance entre deux points exprime la similarité perçue entre eux. Plus la distance est grande, moins ils sont similaires, et inversement.
\par
Contrairement aux \textit{propriétés}, les \bolditalic{concepts} recouvrent plusieurs domaines. Chaque concept est \textit{défini} par une combinaison de régions provenant de différents domaines. Ces régions contiennent \textit{toutes les instances possibles} de ce concept sur leur domaine respectif. Par exemple, le concept de « pomme » est représenté par des régions dans quatre domaines : la couleur (e.g. rouge, vert, jaune), le goût (e.g. variant de sucré à acidulé), la forme (e.g. ronde, ovale, cycloïde), la texture (e.g. croquante, molle). Bien plus que de simples \textit{qualificatifs}, ces régions \textit{définissent} le concept de pomme.
\par
Comme évoqué précédemment, les régions sur ces domaines ne s’identifient pas à des prédicats binaires : une pomme singulière est un point dont les coordonnées reflètent les valeurs précises de cette pomme en termes de couleur, goût, forme, et texture (e.g. une pomme rouge foncé, sucrée, etc.). Une telle décomposition fonctionne en principe pour n’importe quel concept, par exemple « oiseau », englobe des propriétés réparties dans différents domaines tels que la forme, la taille, la couleur, le poids, etc.

Montrons explicitement qu’un concept se \textit{décompose} en \textit{régions} dans différents domaines, et comment ces domaines se décomposent chacun en un petit nombre de dimensions.
La région dans l’espace conceptuel qui représente un concept $\mathcal{C}$ peut être vue comme un sous-ensemble du produit cartésien de $n$ différents domaines :
\[ R(\mathcal{C}) \subset \Delta_1 \times \cdots \times \Delta_n \]
où chaque domaine $\Delta_i$ est généralement défini sur une, deux ou trois dimensions et possède une aire/un volume total \textit{fini}.\footnote{Pour les domaines sensoriels comme la vision et l’audition, ce caractère fini est particulièrement intuitif. Cela signifie que pour une dimension telle que la hauteur tonale (“aigu / grave”), il existe des limites au-delà desquelles nous ne percevons pas les sons comme plus aigus ou plus graves. Il en va de même pour la vision, où des limites similaires existent pour les couleurs ou les intensités lumineuses.}  Par commodité, chaque dimension d’un domaine est identifiée à un intervalle borné dans $\mathbb{R}$. Une instance spécifique du concept $\mathcal{C}$ peut être représentée par un point $\mathbf{x}$ :
\[ \mathbf{x} = \langle x_1, \ldots, x_n \rangle \in R(\mathcal{C}) \]
où chaque $x_i$ représente un point dans le domaine $\Delta_i$, et correspond donc à un vecteur de dimension $m \in \{1, 2, 3\}$ selon le nombre de dimensions dans le domaine $\Delta_i$. En d’autres termes, si $\mathbf{x}$ est une instance de $\mathcal{C}$, la composante $x_i$ de $\mathbf{x}$ situe dans la région associée au concept $\mathcal{C}$ sur le domaine $\Delta_i$. Cette région est strictement inclue dans $\Delta_i$. Formellement, si la région associée au concept $\mathcal{C}$ dans le domaine $m$-dimensionnel $\Delta_i$ est notée $R_i(\mathcal{C})$, on a :
\[ x_i \in R_i(\mathcal{C}) \subset \Delta_i \subset \mathbb{R}^m \]

Ainsi la représentation dans un espace conceptuel est \bolditalic{compositionnelle} : toutes les exemplaires spécifiques d'un concept sont des points\footnote{Biensûr, puisque le grain n'est jamais "infiniment fin", les exemplaires pourraient aussi être représentés par de très petites régions (cf.2.5.2)}  localisés dans chacune de ses régions en fonction de leurs attributs sur chaque dimension.
\par
Notons toutefois que la compositionnalité sur laquelle s’appuie la théorie des espaces conceptuels est \textit{plus faible} que la compositionnalité linguistique, qui postule que la sémantique d’un concept peut être décomposée en unités linguistiques \textit{connues}. En effet, en général, \textit{aucun adjectif connu} ne correspond précisément à la région correspondant à un concept sur un domaine donné. Malgré cela, il est souvent possible d’approximer la région correspondant à ce concept par une union de petites régions.\footnote{Par exemple, notons le domaine des couleurs $\Delta_i$ et la région correspondant au concept de ``citron'' dans le domaine des couleurs par $R_i(\text{\og citron \fg{}})$. La région $R_i(\text{\og citron \fg{}})$ semble pouvoir être approximée par :
\[R_i(\text{citron}) \approx \text{JAUNE} \cup \text{\mbox{VERT CLAIR}}\]

Cependant, plus il y a de nuances dans JAUNE $\cup$ VERT CLAIR qui ne sont jamais instanciées par aucun citron, plus cette approximation est imparfaite. Par ailleurs, pour rendre compte de la \textit{répartition inégale} des citrons entre ces deux couleurs (par exemple, s’ils sont plus souvent jaunes que verts), des contraintes supplémentaires seraient nécessaires, autrement dit $R_i(\text{\og citron \fg{}})$ pourrait ne pas exactement correspondre à la sémantique des adjectifs « jaune » et « vert clair » lorsqu’ils sont utilisés seuls.
\par Il n’en reste pas moins qu’il y a souvent une analogie forte entre la région correspondant à un concept dans un domaine et les régions des adjectifs dans ce même domaine.}

\subsection*{L’équivalence entre les représentations arborescentes et spatiales}

\doublespacing  % Définir l'interligne à 1.5

Dans son \textit{Organon}, Aristote définit le \textit{genre} (\textgreek{g'enos}) comme un \textit{prédicat} commun à plusieurs choses ayant des \textit{différences spécifiques} entre elles.\footnote{ARISTOTE, Organon, Topiques I, 5, 102a31-32.} Ainsi le genre est un attribut essentiel partagé par plusieurs espèces (\textgreek{e>'idos}). Par exemple, « animal » est le genre commun à l’espèce humaine, à l’espèce des chevaux, des oiseaux, etc. Chaque espèce se définit par l’ajout d’une différence spécifique au genre. Dans ce cadre, l’espèce des « lions » peut être définie comme des félins (genre) dotés d’une crinière (différence spécifique). Le genre a donc une plus grande extension, car il inclut plusieurs espèces, tandis que l’espèce nécessite plus de conditions dans sa définition, puisqu’elle contient des déterminations supplémentaires par rapport au genre.\footnote{ARISTOTE, Organon, Topiques IV, 1, 121a10-14.}
\par
La relation « X et Y sont deux \textit{sortes} de Z » (ou de manière équivalente « Z est le \textit{genre} de X et Y ») peut être représentée de deux manières distinctes : soit de manière \bolditalic{arborescente} où X et Y sont deux nœuds inférieurs connectés au nœud supérieur Z, soit de manière \bolditalic{spatiale}, où X et Y sont \textit{inclus} à l’intérieur de Z comme dans une boîte. Les propriétés qui définissent Z sont également attribuées à X et Y. Interprété dans une arborescence, cela veut dire que les nœuds inférieurs \textit{héritent} des propriétés des nœuds supérieurs. Interprété spatialement, cela veut dire qu’une propriété définissant un ensemble est \textit{transmise} à ses sous-ensembles. Ainsi, plus un nœud est \textit{haut} dans l’arborescence, plus les propriétés qui le caractérisent sont \textit{générales}, et inversement. De la même manière, moins les conditions requises pour appartenir à Z sont restrictives et spécifiques, plus l’extension de Z sera \textit{vaste}, et inversement. Ces principes semblent particulièrement bien exemplifiés dans les taxonomies des êtres vivants, ou bien dans celles des objets mathématiques. Par exemple, Nicole et Arnauld (1662) proposent une arborescence hiérarchique entre quelques \textit{concepts géométriques}\footnote{NICOLE et ARNAULT (1992, p.311) utilisent un signe d'accolade pour noter la partition d'un ensemble en sous-ensembles. Ils décrivent ces relations par les notions de genres et d'espèces, par exemple : « le quadrilatère, qui est un genre au regard du parallélogramme et du trapèze, est une espèce au regard de la figure » (1992, p. 53).} :

\begin{forest}
for tree={
  grow=east, % Growth direction is from left to right
  parent anchor=east, child anchor=west, % Anchors
  l sep=20pt, % Level separation
  s sep=25pt, % Sibling separation
  edge={line width=1pt, -} % Line width and style
}
[polygone
  [triangle
    [triangle isocèle]
    [triangle droit
      [triangle droit isocèle]
      [triangle droit irrégulier]
    ]
    [triangle obtus]
  ]
  [quadrangle
    [parallélogramme]
    [trapèze]
  ]
  [pentagone]
]
\end{forest}

Chaque nœud de l'arborescence représente un ensemble d’objets satisfaisant des propriétés
déterminées. Nicole et Arnault précisent que l’ensemble des triangles « peut se diviser selon les côtés, ou selon les angles »\footnote{\textit{Ibid} (1992, p.311).}. La division selon les angles est illustrée dans l’arborescence ci-dessus, tandis que la division selon les côtés aurait produit trois branches : triangles isocèles, équilatéraux et scalènes. La partition dépend de la propriété prise en compte.

\section*{Représenter la sémantique des concepts ordinaires comme des régions :
un défi pour la théorie des espaces conceptuels}


Comme évoqué, les relations hiérarchiques entre ces concepts sont représentables, soit par une \textit{arborescence}, soit de manière \textit{spatiale} où l'extension des concepts est représentée par des boîtes incluses les unes dans les autres. Contrairement à la représentation arborescente, la spatialisation permet de quantifier des mesures comme la \textit{taille} des différentes boîtes ou les \textit{distances} entre elles. Une grande part de la recherche mobilise les espaces conceptuels pour saisir ces avantages : il ne suffit pas d'écrire \og CONCEPT A $\subset$ CONCEPT B \fg mais il faut tenter de déterminer la \textit{taille} de l'extension de A par rapport à celle de B, de déterminer la \textit{distance} séparant les extensions de A et de B. L'ambition de déterminer la \textit{taille} des régions correspondant à l'extension des concepts ainsi que leurs \textit{distances} a reçu un certain succès pour les catégories sensorielles et particulièrement pour les \textit{couleurs}.
\par
Depuis environ une décennie, une partie de la recherche\footnote{Derrac et Schockaert (2015, p.69): "Most approaches represent natural language terms as points or vectors"} dans le cadre des espaces conceptuels s'inspire des \textit{plongements sémantiques} (\textit{word embedding}) dans lesquels les mots sont représentés sous la forme de grands vecteurs, ce qui les rend représentables sous forme de \bolditalic{points} dans des espaces avec plusieurs centaines de dimensions \bolditalic{non interprétables}. Puisque chaque mot est représenté par un vecteur unique, ces modèles permettent de mesurer la distance et la similarité entre deux mots. Selon nous, cette approche diffère de la théorie des espaces conceptuels aux deux égards suivants.
\par
Premièrement, la théorie des espaces conceptuels représente les concepts comme des \bolditalic{régions} qui délimitent l'ensemble des objets qui satisfont les conditions d'appartenance du concept. La représentation avec des points rend moins bien compte de la variabilité et de la flexibilité de la sémantique des concepts concernés. Deuxièmement, les dimensions individuelles qui constituent les mots-vecteurs dans les modèles d’embeddings de mots sont généralement très difficiles à interpréter, tandis que dans les espaces conceptuels, chaque dimension représente une qualité \bolditalic{interprétables}\footnote{BECHBERGER (2023, p. 271): “Interpretable dimensions, however, are a corner stone of the conceptual spaces framework”.}. (\textit{e.g.} pour les \textit{couleurs} : le degré de luminosité, de teinte, de saturation (cf. chap. 1), pour le \textit{goût} : le degré d’acidité, de douceur\footnote{D’autres dimensions peuvent aussi être prises en compte dans le domaine du goût (BOLT et al. 2019, p.168). Bien que degré d'acidité soit généralement une dimension séparée du degré de douceur (sweetness) et qu'elles soient séparées du domaine des couleurs, BECHBERGER (2023, pp. 89-97) formalise la corrélation entre la couleur verte d'une pomme et son acidité, ainsi qu’entre la couleur rouge d'une pomme et sa douceur.}, etc.) Il y a de nombreuses techniques permettant de rapprocher les prolongements sémantiques  avec les espaces conceptuels. Avant de les détailler, les motivations sont données dans le paragraphe suivant.
\par
La première motivation concerne à la fois le problème de l’interprétabilité et de la dimensionalité : est-ce que la difficulté à interpréter les vecteurs des mots dans les prolongements sémantiques est liée au fait qu’ils aient plusieurs centaines de dimensions ? Cette question conduit à une autre : est-ce que seuls les espaces de faible dimensionalité sont interprétables ? En réalité, il convient nuancer l'affirmation selon laquelle les mots sont représentables dans des espaces de faible dimensionalité dans la théorie des espaces conceptuels. En effet, ce sont seulement les \textit{domaines} qui sont de faible dimensionalité. Par exemple, la sémantique d’un adjectif qui réfère à \textit{un seul} domaine sensoriel\footnote{Lorsqu’ils sont conçus comme \textit{propriétés}, les adjectifs sont \textit{restreints} à un seul domaine, dans lequel se trouve leur \textit{sens littéral}. Les  \textit{connotations} possibles de l’adjectif  \og bleu \fg ne sont pas capturées dans ce domaine unique. Par exemple, le fait que l’adjectif \og bleu \fg connote la mer, le ciel, ou le calme implique de sortir du domaine des couleurs. Puisque les \textit{propriétés} ne peuvent être définies que sur \textit{un} domaine, la représentation des connotations ne paraît possible que pour les \textit{concepts}.} (comme "bleu", "amer", "aigu") est généralement représentée dans un espace tridimensionnel. Cependant, pour les mots qui ne réfèrent pas à un seul domaine sensoriel, la visualisation en deux ou trois dimensions n’est possible que pour chaque composant sémantique considéré isolément.
\par
Un exemple fréquemment utilisé est celui du concept de \textit{pomme}. La sémantique de ce concept est représentée sur les domaines de la couleur, de la forme, du goût, de la texture ou encore des spécifications nutritionnelles et biologiques\footnote{Gardenfors (2000, sec. 4.2.1), Fiorini (2014), Bechberger (2023, p.87).}. Les régions sur ces différents domaines sont toutes visualisables : on peut par exemple représenter la région tridimensionnelle contenant toutes les couleurs possibles pour les pommes dans l’espace des couleurs. Cependant, si la \textit{sémantique globale} du concept de pomme implique tous ces domaines simultanément, elle devient difficilement visualisable dans un espace de faible dimensionalité où chaque dimension est interprétable. Cet espace conceptuel du concept de pomme est \textit{fabriqué à la main}\footnote{BECHBERGER (2023, p.46): “Handcrafting a conceptual space usually consists in manually defining the dimensions of the conceptual space based on the available sensors”.}, c'est-à-dire que les propriétés de ce concept sont déterminées \textit{a priori}. En général, les espaces "faits main" ont le mérite de définir le concept d’une manière \textit{interprétable} et \textit{intuitive}, mais ont le désavantage de ne pas avoir de pouvoir prédictif, puisqu’ils ne définissent pas chaque propriété du concept de manière \textit{quantitative}. C’est l’une des raisons pour lesquelles, aujourd’hui, une part de la recherche tente d’obtenir des espaces conceptuels en traitant des données. Expliquons donc les étapes qu’il convient de suivre pour ce faire.

\subsection*{Comment construire un espace conceptuel ?}

\bolditalic{La collecte des données}. Avant de construire un espace conceptuel, il faut tout d’abord obtenir des données sur la similarité des $n$ éléments qu’il s’agit d’y représenter\footnote{Douven et al. (2017, p.690): “The structure of a conceptual space is typically determined on the basis of similarity ratings”.}. Cette collecte de données peut employer aussi bien des méthodes expérimentales que computationnelles : on peut demander à des participants d’estimer la similarité entre toutes les paires possibles avec ces $n$ éléments. Des données de la même forme peuvent être obtenues avec des larges modèles de langage (LLMs)\footnote{MOULLEC et DOUVEN (2024) obtiennent des jugements de similarité de qualité entre différents mammifères en utilisant, notamment, Word2vec, FastText et GPT4.}. On peut également opérer manuellement des \textit{statistiques de co-occurrences} entre certains mots sur des corpus de textes ciblés  pour estimer leur proximité sémantique\footnote{Derrac et Schockaert (2015): "The required similarity degrees are often obtained from (…) the co-occurrence". Pour quantifier la co-occurrence entre deux mots, ils utilisent le PPMI (cf. Appendice 1.3).}. Au total $\frac{n(n-1)}{2}$ jugements de similarité doivent être collectés si on veut représenter $n$ éléments. Les données, qui se présentent donc sous forme de valeur de similarité, sont ensuite rangées dans une matrice de similarité. Cette matrice reflète les jugements de similarité entre ces éléments : chaque cellule $(i,j)$ de la matrice contient la similarité de $i$ avec $j$. Puisque les $n$ lignes et $n$ colonnes de cette matrice correspondent aux mêmes objets, toutes les cellules diagonales $(i,i)$ (pour tout entier \( i \leq n \)) contiennent la valeur de similarité maximale (étant admit que tout objet $i$ est maximalement similaire à lui-même), et les deux moitiés triangulaires de chaque côté de cette diagonale sont symétriques\footnote{SHEPARD (1974): "(…) these measures to be arrayed in the below-diagonal triangular half of an n×n matrix n which the n rows and n columns correspond to the same n objects".} (étant admit que, pour toute paire $(i, j) \in \llbracket 1 ; n \rrbracket^2$, la similarité entre $i$ et $j$ est égale à la similarité entre $j$ et $i$).

\par
\bolditalic{La réduction de la dimensionalité}. Bien qu’il y ait une certaine diversité des \textit{sources} et des \textit{protocoles} pouvant être mis en œuvre pour obtenir ces données de similarité, une \textit{même} technique de réduction de dimensionnalité revient presque systématiquement dans la littérature visant à construire des espaces conceptuels à partir de ces données. Cette technique est le \textit{positionnement multidimensionnel} (\textit{Multidimensional Scaling}, MDS). Le MDS permet de projeter ces éléments dans un espace de plus faible dimensionalité, tout en préservant au mieux les relations de similarité initiales\footnote{Bechberger (2023, p.642): “MDS is capable of successfully compressing this information into very low-dimensional spaces”.}. Plus précisément, il s’agit de minimiser une fonction de coût appelée \textit{Stress}, qui mesure l’écart entre les similarités initiales et les distances dans l'espace projeté. La qualité d’une représentation spatiale obtenue par un MDS est visualisable dans un graphique appelé \textit{diagramme de Shepard}. Plus précisément, un diagramme de Shepard représente la relation entre les similarités initiales dans la matrice et les distances obtenues dans l’espace après application du MDS. Les relations de similarités des données initiales sont figurées sur l’axe des abscisses tandis que les distances entre les paires d’éléments sont sur l’axe des ordonnées. Idéalement, si les distances dans l’espace projeté correspondent parfaitement aux similarités initiales, les points dans un diagramme de Shepard devraient se situer exactement sur une droite monotone décroissante\footnote{Borg et Groenen (2005, p.43; 2013, p.35): "[Points] all lie on a monotonically descending line, as requested by the ordinal MDS model used to scale these data".}. A l’inverse, moins l’espace est fidèle aux dissimilarités initiales, plus ces points s’écarteront de cette droite. En général, les espaces de faible dimension sont plus interprétables, mais risquent d’être moins fidèles aux données. A l’inverse, les espaces de grande dimension sont plus fidèles aux données, mais moins interprétables\footnote{Borg et Groenen (2018): "Increasing the dimensionality of the MDS space always makes it easier to find a solution with a better fit". Si les données de départ sont des jugements de dissimilarité, et que le MDS représente de manière fidèle ces données, les points suivront une droite monotone décroissante dans le diagramme de Shepard}. Les diagrammes de Shepard aident à trouver le meilleur compromis entre ces deux contraintes.

L'application d'un MDS sur une matrice de dissimilarité\footnote{Il est facile de transformer une matrice de similarité en matrice de dissimilarité.} permet d'obtenir un espace de faible dimensionalité, où la distance est identifiée à la dissimilarité, qui généralement qualifié d'\textit{espace de similarité}.
L'application du Multidimensional Scaling (MDS) à une matrice de dissimilarité\footnote{Il est facile de transformer une matrice de similarité en matrice de dissimilarité.} permet de créer un espace de similarité, ayant souvent deux ou trois dimensions, et dans lequel les distances entre les points reflètent les dissimilarités présentes dans la matrice d'origine. Cependant, selon nous, l'espace de similarité obtenu à la suite d'un MDS mérite encore d'être enrichi avant d'être qualifié d'espace conceptuel. La première raison tient à ce que, dans un espace de similarité, chaque concept et chaque propriété sont représentés par un \textit{point}, tandis que dans un espace conceptuel, ils sont représentés par des \textit{régions}. La seconde raison tient à ce que, dans un espace de similarité obtenu par MDS, les dimensions peuvent rester abstraites et difficiles à interpréter. Bien qu'il soit souhaitable que chaque dimension ait une signification claire et identifiable, représentant un attribut ou une caractéristique conceptuelle, une part de la recherche\footnote{C'est par exemple le cas de I. Douven, mais aussi de J-L. Dessalles (2018)} tient l'interprétabilité des dimensions comme une condition non nécessaire pour parler d'espace conceptuel. De plus, la contrainte d’interprétabilité rentre parfois en tension avec l’exigence de performances des prédictions du modèle. Parfois mais pas toujours. Ainsi, I. Douven avant un espace conceptuel des couleurs partitionné en région qui allie à la fois faible dimensionalité, interprétabilité des dimensions et performances prédictives des jugements humains sur les couleurs\footnote{Ces jugements sont notamment les jugements de similarité, la catégorisation, l'induction. Cette performance prédictive doit beaucoup à l’amélioration des espaces colorimétriques, dans lesquels les nuances de couleur sont organisées de manière fidèle à la perception humaine.}. Selon Douven et Gärdenfors (2019), cette application est un \textit{succès}, car les régions correspondant aux catégories des couleurs semblent suivre certaines contraintes d’optimalité (\textit{c.f.} 1.4.4). L’application de la théorie des espaces conceptuels aux couleurs s'inscrit dans la lignée des travaux de Shepard (1974), qui cherchaient déjà à placer les couleurs dans un « espace psychologique » \textit{interprétable}. L’application de la théorie des espaces conceptuels aux couleurs hérite donc de l'ambition de Shepard d’extraire des \textit{lois générales} des données empiriques, où les variables sont interprétées par des objets de la psychologie cognitive, mais dont les relations relèvent des mathématiques. Un exemple clé est celui du jugement de similarité comme fonction. A partir des données sur les jugements de similarité entre couleurs, Shepard infère une \textit{fonction} posant que la similarité entre deux couleurs est une fonction exponentielle décroissante de leur distance dans « l’espace psychologique des couleurs ». La loi de Shepard stipule que le degré de similarité entre $x$ et $y$ est donné par une fonction $Sim(x,y)$ définie comme suit :
\[
{Sim}(x,y) = e^{-c \cdot \mathcal{D}(x,y)}
\]
où $\mathcal{D}(x,y)$ est la distance entre $x$ et $y$ dans l’espace psychologique, et $c>0$ est un paramètre de sensibilité qui contrôle la \textit{vitesse} de décroissance de la similarité en fonction de la distance. L’ambition de Shepard est d’utiliser une telle fonction pour des stimuli de différentes natures (visuels, auditifs, etc.) et pour différentes espèces (humains, singes, oiseaux, etc.)\footnote{Shepard (1987, p.1319): “The data are from a number of researchers, who tested both visual and auditory stimuli, and both human and animal subjects. Yet, in every case, the decrease of generalization with psychological distance is monotonic, generally concave upward, and more or less approximates a simple exponential decay function.”}. Cette fonction est très largement utilisée dans la recherche dans le cadre de la théorie des espaces conceptuels\footnote{Gardenfors (2000, sec. 1.6.5) ; Osta-Velez (2020, p.82) ; Bechberger (2023, p.14).}. L’une des idées que Gärdenfors reprend à Shepard est que plus deux points sont \textit{éloignés} dans l’espace de similarité, moins il est probable qu’ils appartiennent à la \textit{même} catégorie, et inversement, plus ils sont proches, plus il est probable qu’il existe une catégorie dont l’extension les contient tous les deux\footnote{Poth (2019, p.9) : “For a set of stimuli, $i$ and $j$ , the empirical probability of an organism to generalise a type of behaviour towards $j$ upon having observed $i$ is a monotone and exponentially decreasing function of the distance between $i$ and $j$ in a continuous psychological similarity space”.}. Cette hypothèse est cruciale parce qu’elle suggère que les points appartenant à une même catégorie ont tendance à apparaître comme des  \textit{clusters} dans un espace de similarité. Si une catégorie spécifique regroupe des éléments homogènes et similaires entre eux, ceux-ci devraient prendre la forme d’un cluster dense et séparable du reste des points dans l’espace de similarité\footnote{BORG et GROENEN (2005, p.104):"(…) a cluster is a particular region whose points are all closer to each other than to any point in some other region. This makes the points in a cluster look relatively densely packed, with "empty" space around the cluster. For regions, such a requirement generally is not relevant".}. A l’inverse, plus une catégorie est hétérogène, plus le cluster des points représentant ses éléments sera \textit{diffus}, plus les points seront \textit{étalés} dans l’espace de similarité. C’est en combinant cette hypothèse à la \textit{théorie du prototype} que Gärdenfors a pu proposer une technique géométrique simple et élégante pour construire des régions à partir des points (\textit{c.f.} 1.4.4).
\par
L’un des avantages de l’hypothèse de Shepard est qu’elle peut être \textit{expérimentalement testée} avec n’importe quels éléments. Il convient de commencer par collecter des jugements de similarité entre une liste de mots prédéfinis, par exemple "chêne", "peuplier", "érable", "abricotier", "bananier", "cerisier", puis d’appliquer un MDS pour obtenir un espace de similarité, et enfin il faut vérifier si les points correspondant aux mots d’une même catégorie, en l’occurrence celle des arbres fruitiers, forment un cluster \textit{dense et petit}. On peut employer diverses pour rendre compte de la proximité entre les points dans cet espace de similarité\footnote{BORG et GROENEN (2018, p.104): "checking to what extent items constructed to measure the same construct appear homogeneous. This analysis can be made easier by drawing convex hulls around items that belong to the same category"}.
\par
Pour rendre cela plus concret, donnons un autre exemple : on peut déterminer une liste d’animaux, qui sont liés à des caractéristiques, comme leurs habitats respectifs (terrestre, aquatique, aérien), leurs tailles (petit, moyen, grand). Les données issues des jugements de similarité sur ces animaux peuvent être \textit{projetées} dans un espace de similarité en appliquant un MDS. Puis il s’agit d’interpréter cet espace \textit{à partir} de la liste des caractéristiques initialement établie. C’est précisément ce que proposent Douven et al. (2023), avec l’espace de similarité suivant :


\begin{figure}[h!]
\centering
\begin{tikzpicture}
\begin{axis}[
    width=95.28mm,
    height=95.28mm,
    xmin=0, xmax=95.28,
    ymin=0, ymax=95.28,
    axis on top,
    y dir=reverse,
    xtick=\empty,
    ytick=\empty,
]
    \node at (axis cs:79.26,44.85) {camel};
    \node at (axis cs:38.02,26.66) {cat};
    \node at (axis cs:62.72,33.04) {chimpanzee};
    \node at (axis cs:85.60,66.20) {cow};
    \node at (axis cs:37.97,38.26) {dog};
    \node at (axis cs:77.28,54.73) {donkey};
    \node at (axis cs:47.73,30.98) {fox};
    \node at (axis cs:76.48,39.04) {giraffe};
    \node at (axis cs:79.91,73.25) {goat};
    \node at (axis cs:52.06,27.68) {gorilla};
    \node at (axis cs:85.08,49.19) {horse};
    \node at (axis cs:44.85,4.40) {lion};
    \node at (axis cs:19.59,88.18) {mouse};
    \node at (axis cs:69.39,71.54) {pig};
    \node at (axis cs:28.52,83.79) {rabbit};
    \node at (axis cs:30.93,93.17) {rat};
    \node at (axis cs:83.94,69.74) {sheep};
    \node at (axis cs:40.54,11.28) {tiger};
    \node at (axis cs:46.16,24.65) {wolf};
    \node at (axis cs:84.98,37.03) {zebra};
\end{axis}
\end{tikzpicture}
\caption{Espace de similarité contenant 20 mammifères. Extrait de Douven et al. (2023)} 
\end{figure}

L’axe $y$ semble corrélé à la férocité des vingt mammifères\footnote{Henley (1969, p.180): “[the dimension] seems to be characterizable as one of mildness vs. ferocity”.} et l’axe $x$ semble corrélé à leurs tailles. On peut essayer de donner du sens aux différents clusters, en identifiant celui du bas comme correspondant aux petits animaux, celui à gauche comme correspondant aux herbivores, et celui du haut aux carnivores\footnote{Henley (1969, pp.180-181): "Again there is the grouping into small animals (rabbit and mouse), herbivores (cow, deer, horse, goat, sheep) and other carnivores (bear, lion, dog, cat)".}.

Dans ce mémoire, nous explorons la question de savoir si les espaces conceptuels sont adaptés \textit{uniquement} aux catégories sensorielles, et difficilement applicables aux catégories dont la signification englobe une plus grande variété de domaines. En effet, l’utilisation des \textit{domaines}, c’est-à-dire d’ensembles de dimensions indissociables, paraît  \textit{prima facie} particulièrement adaptée pour les catégories sensorielles. D’un côté, Gärdenfors (2014) affirme que les dimensions des espaces conceptuels sont étroitement liées à ce qui est produit par les récepteurs sensoriels\footnote{Gardenfors (2014): “These dimensions are closely connected to what is produced by our sensory receptors”}, de l’autre côté, il écrit qu’il existe aussi des dimensions qualitatives abstraites et non sensorielles\footnote{“there also exist quality dimensions that are of an abstract, non sensory character” (2015).}. Les recherches ont souvent porté sur la modalité des \textit{couleurs} définie par les dimensions de la teinte, de l'intensité et de la luminosité. D’autres recherches l’ont appliqué à la catégorie des \textit{sons} en utilisant des dimensions telles que la hauteur et l'intensité\footnote{Bolt et al. (2019, chap.9).}. D’autres encore l’ont appliqué à la modalité du goût avec des dimensions telles que le sucré, l'acide, l'amer et le salé\footnote{Bechberger (2023, 75-78) note les dimensions du goût par “TASTE = 〈sweet ; sour ; bitter ; salt〉"}.
\par
À notre connaissance, aucun modèle ne permet aujourd’hui de représenter la sémantique des concepts du langage naturel en combinant (i) \textit{interprétabilité} des dimensions, (ii) représentation des concepts comme \textit{régions} et (iii) performances prédictives.


\end{document}

\end{document}
